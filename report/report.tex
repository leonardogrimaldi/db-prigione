\documentclass[a4paper,12pt]{report}
\usepackage{alltt, fancyvrb, url}
\usepackage{graphicx}
\usepackage[utf8]{inputenc}
\usepackage{float}
\usepackage{hyperref}
\usepackage{minted}

% Questo commentalo se vuoi scrivere in inglese.
\usepackage[italian]{babel}

\usepackage[italian]{cleveref}
\title{Relazione dell'elaborato di Basi di Dati
    \\ Sistema di gestione penitenziario}

\author{Leonardo Grimaldi}
\date{\today}   
\begin{document}
\maketitle
\tableofcontents
\chapter{Analisi}
\section{Introduzione}
Viene commissionata da un ente governativo la realizzazione di un software gestionale per una casa circondariale che faciliti il tracciamento di detenuti e loro spostamenti.
\section{Intervista}
Si chiede di realizzare un portale che consenta di gestire e storicizzare varie operazioni comuni di un carcere. Per i detenuti in arrivo si vogliono memorizzare gli estremi della persona.
%
I dati richiesti sono:
\begin{itemize}
    \item Nome, cognome, data di nascita, il numero della carta d'identità, altezza
\end{itemize}
Il carcere gestisce solamente detenuti italiani maggiorenni in possesso di carta d'identità quindi non occorre gestire il caso in cui essa non sia presente.
%
Un detenuto può essere rilasciato e rientrare nel carcere, ma anche decedere durante la sua permanenza.
%
Ai detenuti sono assegnate delle celle letto in base alla disponibilità.
%
\\Nel corso della loro permanenza le assegnazioni possono subire variazioni e si dovrà quindi tenere traccia degli spostamenti.
%
Questo include la data e ora di uscita e in quale cella è avvenuto lo spostamento.
%
\\All'interno della prigione sono presenti anche celle mediche e solitarie all'interno delle quali il prigioniero può risiedere temporaneamente.
%
Ogni cella appartiene a un blocco che viene pattugliato da una o più guardie.
%
I turni di pattuglia sono assegnati in base a un orario prestabilito in cui ogni giorno della settimana è formato da 3 turni:
\begin{itemize}
    \item Mattina: 06:00 - 14:00
    \item Pomeriggio/sera: 14:00 - 22:00
    \item Notte: 22:00 - 06:00 (del giorno successivo)
\end{itemize}
La guardia lavorerà quindi per 8 ore al giorno con una pausa intermedia di 30 minuti e fine turno di 30 minuti.
%
Le pause e i cambi di turno non verranno gestiti dal database ai fini di copertura dell'orario, ma si suppone che vi sia una guardia di riserva che subentra temporaneamente.
%
\\Il personale del carcere è formato quindi da guardie, ma anche da amministratori e di entrambi si vuole memorizzare: il nome, cognome, data di nascita, sesso e codice fiscale.
%
Gli amministratori sono le persone che hanno accesso al sistema gestionale e possono essere anche le guardie stesse.
%
Sia le guardie che gli amministratori possiedeno un badge che li identifica univocamente all'interno della struttura.
%
Di loro si vuole memorizzare inoltre: 
\begin{itemize}
    \item Nome, cognome, codice fiscale e sesso.
\end{itemize}
Il sistema non dovrà gestire la storicizzazione del personale e dei cambi di orario.  
\section{Analisi dei requisiti}
Realizzare uno script Python per monitorare lo stato di una rete, controllando la disponibilità di uno o più host tramite il protocollo ICMP (ping).
%
Lo script deve consentire all'utente di specificare gli indirizzi IP degli host da monitorare e deve visualizzare lo stato (online/offline) di ciascun host.
\end{document}