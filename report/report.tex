\documentclass[a4paper,12pt]{report}
\usepackage{alltt, fancyvrb, url}
\usepackage{graphicx}
\usepackage[utf8]{inputenc}
\usepackage{float}
\usepackage{hyperref}
\usepackage{minted}

% Questo commentalo se vuoi scrivere in inglese.
\usepackage[italian]{babel}

\usepackage[italian]{cleveref}
\title{Relazione dell'elaborato di Basi di Dati
    \\ Sistema di gestione penitenziario}

\author{Leonardo Grimaldi}
\date{\today}   
\begin{document}
\maketitle
\tableofcontents
\chapter{Analisi}
\section{Introduzione}
Viene commissionata da un ente governativo la realizzazione di un software gestionale per una casa circondariale che faciliti il tracciamento di detenuti e loro spostamenti.
\section{Intervista}
Si chiede di realizzare un portale dove poter inserire i detenuti in arrivo e memorizzare gli estremi della persona.
%
I dati richiesti sono:
\begin{itemize}
    \item Nome, cognome, data di nascita, il numero della carta d'identità, altezza
\end{itemize}
Nel caso in cui il prigioniero non abbia la C.I. verrà usato il passaporto oppure l'impronta digitale.
%
All'interno del carcere è presente un lettore di impronte digitali ed è necessario salvare l'impronte di ognuno.
%
Ai detenuti sono assegnate delle celle letto in base alla disponibilità.
%
\\Nel corso della loro permanenza le assegnazioni possono subire variazioni e si dovrà quindi tenere traccia degli spostamenti.
%
Questo include la data e ora di uscita e in quale cella è avvenuto lo spostamento.
%



\section{Analisi dei requisiti}

Realizzare uno script Python per monitorare lo stato di una rete, controllando la disponibilità di uno o più host tramite il protocollo ICMP (ping).
%
Lo script deve consentire all'utente di specificare gli indirizzi IP degli host da monitorare e deve visualizzare lo stato (online/offline) di ciascun host.
\end{document}